\documentclass{article}
\usepackage{../fasy-hw}

%% UPDATE these variables:
\renewcommand{\hwnum}{1}
\title{Advanced Algorithms, Homework \hwnum}
\author{\todo{Your Name Here}}
\collab{\todo{list your collaborators here}}
\date{due: 13 September 2021}

\begin{document}

\maketitle

This homework assignment should be
submitted as a single PDF file both to D2L and to Gradescope.

General homework expectations:
\begin{itemize}
    \item Homework should be typeset using LaTex.
    \item Answers should be in complete sentences and proofread.
    \item You will not plagiarize, nor will you share your written solutions
        with classmates.
    \item List collaborators at the start of each question using the
        \texttt{collab} command.
    \item Put your answers where the \texttt{todo} command currently is (and
        remove the \texttt{todo}, but not the word \texttt{Answer}).
    \item If you are asked to come up with an algorithm, you are
        expected to give an algorithm that beats the brute force (and, if possible, of
        optimal time complexity). With your algorithm, please provide the following:
        \begin{itemize}
            \item \emph{What}: A prose explanation of the problem and the algorithm,
                including a description of the input/output.
            \item \emph{How}: Describe how the algorithm works, including giving
                psuedocode for it.  Be sure to reference the pseudocode
                from within the prose explanation.
            \item \emph{How Fast}: Runtime, along with justification.  (Or, in the
                extreme, a proof of termination).
            \item \emph{Why}: Statement of the loop invariant for each loop, or
                recursion invariant for each recursive function.
        \end{itemize}
\end{itemize}



%%%%%%%%%%%%%%%%%%%%%%%%%%%%%%%%%%%%%%%%%%%%%%%%%%%%%%%%%%%%%%%%%%%%%%%%%%%%%%
\collab{N/A}
\nextprob{Groups}

Work in a group of size $\geq 2$.  Explain your strategy for working in a group.

\paragraph{Answer}
\todo{}

%%%%%%%%%%%%%%%%%%%%%%%%%%%%%%%%%%%%%%%%%%%%%%%%%%%%%%%%%%%%%%%%%%%%%%%%%%%%%%
\collab{\todo{}}
\nextprob{More Basics}

\begin{enumerate}

    \item Use the definition of big-Theta notation to prove that $f(x)=4n-5$
        is $\Theta(n+1)$.

        \paragraph{Answer}
        \todo{}

    \item Use induction to prove that, for all $n \in \N$, the complete graph on
        $n$ vertices (denoted by the symbol~$K_n$) has $\frac{n(n-1)}{2}$~edges.

        \paragraph{Answer}
        \todo{}

    \item What is the negation of the following statement: for all integers $n
        \in \Z$, $n$ is prime.

        \paragraph{Answer}
        \todo{}

\end{enumerate}


%%%%%%%%%%%%%%%%%%%%%%%%%%%%%%%%%%%%%%%%%%%%%%%%%%%%%%%%%%%%%%%%%%%%%%%%%%%%%%
\collab{\todo{}}
\nextprob{Sorting before Searching?}
Let $A$ be an array of $n$ comparable objects.  We do not know if $A$ is sorted
or not.

\begin{enumerate}
    \item To answer the question \emph{is item $x$ in $A$?}, should we
        sort the array first?  Why or why not?

        \paragraph{Answer}
        \todo{}

    \item Suppose we have $k$ elements that we want to find in $A$. Does this
        change your answer? Why or why~not?

        \paragraph{Answer}
        \todo{}

\end{enumerate}

%%%%%%%%%%%%%%%%%%%%%%%%%%%%%%%%%%%%%%%%%%%%%%%%%%%%%%%%%%%%%%%%%%%%%%%%%%%%%%
\collab{\todo{}}
\nextprob{Recurrence Relations}
Consider the function $T \colon \N \to \N$ defined by
$$T(n) = \begin{cases}
            1        & n=1\\
            T(n-1)+1 & n>1.
         \end{cases}
$$
What is the closed form and asymptotic form of this recursion?  For the
closed form, use induction to prove that it is correct.  For the asymptotic
form, use the definition of big-Theta to justify.

\paragraph{Answer}
\todo{}


%%%%%%%%%%%%%%%%%%%%%%%%%%%%%%%%%%%%%%%%%%%%%%%%%%%%%%%%%%%%%%%%%%%%%%%%%%%%%%
\collab{\todo{}}
\nextprob{More Recurrence Relations}

What is the asymptotic form of the following recurrence
relations? (Show work for partial credit, but full justification is not required
on this question).
Let $T \colon \N \to N$ be defined by $T(1)=1$ and, for $n>1$,
\begin{enumerate}
    \item $T(n) = 16 T(n/4) + n$
        \paragraph{Answer} \todo{}
    \item $T(n) = 2 T(n/2) + n \log{n}$
        \paragraph{Answer} \todo{}
    \item $T(n) = 6 T(n/3) + n^2 \log{n}$
        \paragraph{Answer} \todo{}
    \item $T(n) = 4 T(n/2) + n^2$
        \paragraph{Answer} \todo{}
    \item $T(n) = 9 T(n/3) + n$
        \paragraph{Answer} \todo{}
\end{enumerate}


%%%%%%%%%%%%%%%%%%%%%%%%%%%%%%%%%%%%%%%%%%%%%%%%%%%%%%%%%%%%%%%%%%%%%%%%%%%%%%
\collab{\todo{}}
\nextprob{Pancakes}

Chapter 1, Problem 9, parts (a) and (b).

\paragraph{Answer, Part (a)}

\todo{}

\paragraph{Answer, Part (b)}

\todo{}

%%%%%%%%%%%%%%%%%%%%%%%%%%%%%%%%%%%%%%%%%%%%%%%%%%%%%%%%%%%%%%%%%%%%%%%%%%%%%%
\collab{\todo{}}
\nextprob{Largest Complete Subtree}

Chapter 1, Problem 37.  For this problem, please provide a full proof of
correctness (i.e., provide a proof for the recursion invariant).

\paragraph{Answer}
\todo{}



\end{document}

\documentclass{article}
\usepackage{../fasy-hw}

%% UPDATE these variables:
\renewcommand{\hwnum}{7}
\title{Advanced Algorithms, Homework \hwnum}
\author{\todo{Your Name Here}}
\collab{\todo{list your collaborators here}}
\date{due: Monday, 8 December 2021}

\begin{document}

\maketitle

This homework assignment should be
submitted as a single PDF file both to D2L and to Gradescope.

General homework expectations:
\begin{itemize}
    \item Homework should be typeset using LaTex.
    \item Answers should be in complete sentences and proofread.
    \item You will not plagiarize, nor will you share your written solutions
        with classmates.
    \item List collaborators at the start of each question using the
        \texttt{collab} command.
    \item Put your answers where the \texttt{todo} command currently is (and
        remove the \texttt{todo}, but not the word \texttt{Answer}).
    \item If you are asked to come up with an algorithm, you are
        expected to give an algorithm that beats the brute force (and, if possible, of
        optimal time complexity). With your algorithm, please provide the following:
        \begin{itemize}
            \item \emph{What}: A prose explanation of the problem and the algorithm,
                including a description of the input/output.
            \item \emph{How}: Describe how the algorithm works, including giving
                psuedocode for it.  Be sure to reference the pseudocode
                from within the prose explanation.
            \item \emph{How Fast}: Runtime, along with justification.  (Or, in the
                extreme, a proof of termination).
            \item \emph{Why}: Statement of the loop invariant for each loop, or
                recursion invariant for each recursive function.
        \end{itemize}
\end{itemize}


\collab{\todo{}}
\nextprob{}

The \texttt{rand()} function in the standard C library returns a
uniformly random number in \texttt{[0,RANDMAX-1]}. Does \texttt{rand}()$\mod n$
generate a number uniformly distributed in $[0,n-1]$? (Prove or disprove).

% Note I: This is the second variant in EPI 5.12.



\collab{\todo{}}
\nextprob{}

Algorithms where we use randomization to find a deterministic answer are known
as Las Vegas algorithms.  Monte Carlo algorithms also use randomization, but
might not always give the right answer; however, they either have a high
probability of being correct or close to correct.

\begin{enumerate}[(a)]
    \item Give a Monte Carlo algorithm to estimate~$\pi$.
    \item Let $n$ be the number of random numbers used by your algorithm.
        Explain why as $n \to \infty$, the expectation of the output for your
        algorithm is~$\pi$.
    \item Implement this algorithm and plot a line graph of
        the values returned for at least $10$ values of~$n$.
\end{enumerate}

Note: Assume that there is a function \texttt{randReal}$[a,b]$ that returns a random
real number between $a$ and $b$, iid from the uniform distribution over the
interval $[a,b]$.



\collab{\todo{}}
\nextprob{}

Choose an algorithm that you analyzed on a homework in this class (can be this
HW or a previous one).  Suppose you are a journalist writing about this
break-through algorithm and write a one-page summary of the algorithm for a
general audience.  Describing the problem that this algorithm solves and the
applications of the problem should be highlighted (feel free to do some
research).  Detail of the algorithm and proofs of correctness or runtime should
be only given at a very high level.

\paragraph{Answer}
\todo{answer here}



\collab{\todo{}}
\nextprob{Removing an Edge}

Chapter 8, Question 4, Part(a)


\collab{\todo{}}
\nextprob{}

Chapter 10, Question 4, (Opposing Edges)

\paragraph{Answer}
\todo{answer here}

\collab{\todo{}}
\nextprob{}

Chapter 11, Question 6, (Mini-Golf)

\paragraph{Answer}
\todo{answer here}






\end{document}

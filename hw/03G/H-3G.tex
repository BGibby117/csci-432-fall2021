\documentclass{article}
\usepackage{../fasy-hw}

%% UPDATE these variables:
\renewcommand{\hwnum}{1}
\title{Advanced Algorithms, Homework \hwnum}
\author{\todo{Your Name Here}}
\collab{\todo{list your collaborators here}}
\date{due: Wednesday, 13 October 2021}

\begin{document}

\maketitle

This homework assignment should be
submitted as a single PDF file both to D2L and to Gradescope.

General homework expectations:
\begin{itemize}
    \item Homework should be typeset using LaTex.
    \item Answers should be in complete sentences and proofread.
    \item You will not plagiarize, nor will you share your written solutions
        with classmates.
    \item List collaborators at the start of each question using the
        \texttt{collab} command.
    \item Put your answers where the \texttt{todo} command currently is (and
        remove the \texttt{todo}, but not the word \texttt{Answer}).
    \item If you are asked to come up with an algorithm, you are
        expected to give an algorithm that beats the brute force (and, if possible, of
        optimal time complexity). With your algorithm, please provide the following:
        \begin{itemize}
            \item \emph{What}: A prose explanation of the problem and the algorithm,
                including a description of the input/output.
            \item \emph{How}: Describe how the algorithm works, including giving
                psuedocode for it.  Be sure to reference the pseudocode
                from within the prose explanation.
            \item \emph{How Fast}: Runtime, along with justification.  (Or, in the
                extreme, a proof of termination).
            \item \emph{Why}: Statement of the loop invariant for each loop, or
                recursion invariant for each recursive function.
        \end{itemize}
\end{itemize}

{\bf
This homework can be submitted as a group of size $n \geq 1$.
}

%%%%%%%%%%%%%%%%%%%%%%%%%%%%%%%%%%%%%%%%%%%%%%%%%%%%%%%%%%%%%%%%%%%%%%%%%%%%%%
\collab{\todo{}}
\nextprob{Sugar Packet Game}

In Section 2.2 of the textbook, Dr.~Erickson describes a two-player game.
He states, ``It's not hard to prove that
as long as there are tokens on the board, at least one player can legally
move.''  Prove this statement is correct.

\paragraph{Answer}

\todo{}


%%%%%%%%%%%%%%%%%%%%%%%%%%%%%%%%%%%%%%%%%%%%%%%%%%%%%%%%%%%%%%%%%%%%%%%%%%%%%%
\collab{\todo{}}
\nextprob{Repeated pattern}

Chapter 3, Problem 26.  Only one algorithm is needed.  Bonus points for a
solution that beats $O(n^5)$ though!

\paragraph{Answer}

\todo{}

%%%%%%%%%%%%%%%%%%%%%%%%%%%%%%%%%%%%%%%%%%%%%%%%%%%%%%%%%%%%%%%%%%%%%%%%%%%%%%
\collab{\todo{}}
\nextprob{Greedy Algorithm}

What is the loop invariant of the for loop for the \textsc{GreedySchedule}
algorithm in Section 4.2 of the textbook?  Please state all relevant statements
(post-condition, pre-condition, loop invariant, and loop guard).  It may be
helpful to rewrite this for loop as a while loop and work with that.  Prove that
your loop invariant is correct.

\paragraph{Answer}

\todo{}

\end{document}

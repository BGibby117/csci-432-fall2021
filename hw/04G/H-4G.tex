\documentclass{article}
\usepackage{../fasy-hw}

%% UPDATE these variables:
\renewcommand{\hwnum}{4}
\title{Advanced Algorithms, Homework \hwnum}
\author{\todo{Your Name Here}}
\collab{\todo{list your collaborators here}}
\date{due: Monday, 25 October 2021}

\begin{document}

\maketitle

This homework assignment should be
submitted as a single PDF file both to D2L and to Gradescope.

General homework expectations:
\begin{itemize}
    \item Homework should be typeset using LaTex.
    \item Answers should be in complete sentences and proofread.
    \item You will not plagiarize, nor will you share your written solutions
        with classmates.
    \item List collaborators at the start of each question using the
        \texttt{collab} command.
    \item Put your answers where the \texttt{todo} command currently is (and
        remove the \texttt{todo}, but not the word \texttt{Answer}).
    \item If you are asked to come up with an algorithm, you are
        expected to give an algorithm that beats the brute force (and, if possible, of
        optimal time complexity). With your algorithm, please provide the following:
        \begin{itemize}
            \item \emph{What}: A prose explanation of the problem and the algorithm,
                including a description of the input/output.
            \item \emph{How}: Describe how the algorithm works, including giving
                psuedocode for it.  Be sure to reference the pseudocode
                from within the prose explanation.
            \item \emph{How Fast}: Runtime, along with justification.  (Or, in the
                extreme, a proof of termination).
            \item \emph{Why}: Statement of the loop invariant for each loop, or
                recursion invariant for each recursive function.
        \end{itemize}
\end{itemize}

{\bf
This homework can be submitted as a group of size $3$ or $4$.  This group will
be your project.
}

%%%%%%%%%%%%%%%%%%%%%%%%%%%%%%%%%%%%%%%%%%%%%%%%%%%%%%%%%%%%%%%%%%%%%%%%%%%%%%
\collab{\todo{}}
\nextprob{Class Project}

Answer the following questions relating to your group project for this class.
The answers provided here are not binding for your course project, but are meant
to help you think about the project and start planning for it.

\begin{enumerate}

    \item For each team member, provide one strength and one weakness, as it
        pertains to the course project.  (Note: everyone has strengths and
        weaknesses.  The weaknesses that you are able to identify are those that
        you are able to work on addressing).

        \paragraph{Answer}
        \todo{}

    \item Suppose your group goes for the research option.  Find two ``recent"
        algorithms that were published in a journal or a conference.  Describe
        the problem that these algorithms solve.

        \paragraph{Answer}
        \todo{}

    \item Suppose your group goes for the implementation option. Describe two
        methods that you can use to compare implementations.

        \paragraph{Answer}
        \todo{}

\end{enumerate}


%%%%%%%%%%%%%%%%%%%%%%%%%%%%%%%%%%%%%%%%%%%%%%%%%%%%%%%%%%%%%%%%%%%%%%%%%%%%%%
\collab{\todo{}}
\nextprob{Different Sources}

For this problem, choose either the Edit distance algorithm (Section 3.7).
Find three additional sources (including the
textbook) that describe
and analyze the same algorithm. In one to two pages, describe the similarities
and differences in the presentation and analysis of the algorithms. (Note:
proper citations are expected).

\paragraph{Answer}
\todo{}

%%%%%%%%%%%%%%%%%%%%%%%%%%%%%%%%%%%%%%%%%%%%%%%%%%%%%%%%%%%%%%%%%%%%%%%%%%%%%%
\collab{\todo{}}
\nextprob{Dynamic Programming}

Suppose someone poses a problem to you, and you have a hunch that it can be
solved with a dynamic program.  Describe, in your own words, the steps you will
take to work through finding a solution to the problem.  If it helps, you can
choose an example to illustrate working through the process.

\paragraph{Answer}
\todo{}


%%%%%%%%%%%%%%%%%%%%%%%%%%%%%%%%%%%%%%%%%%%%%%%%%%%%%%%%%%%%%%%%%%%%%%%%%%%%%%
\collab{\todo{}}
\nextprob{Greedy Schedule}

Chapter 4, Question 1 (Greedy Schedule).  I encourage you to think through all 9
alternative schedules.  However, you only need to hand in two:
\begin{enumerate}
    \item Choose one alternate strategy that
        works, and prove that it works.

    \paragraph{Answer}
    \todo{}

    \item Choose one alternate strategy that does not work, and give a
        counter-example.

    \paragraph{Answer}
    \todo{}

\end{enumerate}

%%%%%%%%%%%%%%%%%%%%%%%%%%%%%%%%%%%%%%%%%%%%%%%%%%%%%%%%%%%%%%%%%%%%%%%%%%%%%%
\collab{\todo{}}
\nextprob{Real Graphs}

Describe a ``real-life'' problem that can be modeled as:

\begin{enumerate}
    \item An undirected graph.

        \paragraph{Answer}
        \todo{}

    \item A directed, weighted graph.

        \paragraph{Answer}
        \todo{}

    \item A tree.

        \paragraph{Answer}
        \todo{}

    \item A forest.

        \paragraph{Answer}
        \todo{}

\end{enumerate}

\end{document}
